\documentclass[compress]{beamer}
\usepackage{ifthen,verbatim}

\title{Muon Alignment Workflow \\ with the HIP Algorithm}
\author{Jim Pivarski, Alexei Safonov, Karoly Banicz}
\institute{Texas A\&M University}
\date{25 January, 2008}

\newcommand{\isnote}{}
\xdefinecolor{lightyellow}{rgb}{1.,1.,0.25}
\xdefinecolor{darkblue}{rgb}{0.1,0.1,0.7}

%% Uncomment this to get annotations
%% \def\notes{\addtocounter{page}{-1}
%%            \renewcommand{\isnote}{*}
%% 	   \beamertemplateshadingbackground{lightyellow}{white}
%%            \begin{frame}
%%            \frametitle{Notes for the previous page (page \insertpagenumber)}
%%            \itemize}
%% \def\endnotes{\enditemize
%% 	      \end{frame}
%%               \beamertemplateshadingbackground{white}{white}
%%               \renewcommand{\isnote}{}}

%% Uncomment this to not get annotations
\def\notes{\comment}
\def\endnotes{\endcomment}

\setbeamertemplate{navigation symbols}{}
\setbeamertemplate{headline}{\includegraphics[height=1 cm]{../cmslogo} \hspace{0.1 cm} \includegraphics[height=1 cm]{../tamulogo} \hfill
\begin{minipage}{5.5 cm}
\vspace{-0.75 cm} \small
\begin{center}
\ifthenelse{\equal{\insertpagenumber}{1}}{}{\textcolor{blue}{\insertsection}}
\end{center}
\end{minipage} \hfill
\begin{minipage}{4.5 cm}
\vspace{-0.75 cm} \small
\begin{flushright}
\ifthenelse{\equal{\insertpagenumber}{1}}{}{Jim Pivarski \hspace{0.5 cm} \insertpagenumber\isnote/\pageref{numpages}}
\end{flushright}
\end{minipage}\mbox{\hspace{0.2 cm}}}

\begin{document}
\frame{\titlepage}

%% \begin{notes}
%% \item This is the annotated version of my talk.
%% \item If you want the version that I am presenting, download the one
%% labeled ``slides'' on Indico (or just ignore these yellow pages).
%% \item The annotated version is provided for extra detail and a written
%% record of comments that I intend to make orally.
%% \item Yellow notes refer to the content on the {\it previous} page.
%% \item All other slides are identical for the two versions.
%% \end{notes}

\begin{frame}
\frametitle{Muon-HIP alignment procedures (in chronological order)}
\begin{enumerate}\setlength{\itemsep}{0.35 cm}
\item \textcolor{darkblue}{Cosmics/beam-halo alignment}
\begin{itemize}
\item Aligning chambers relative to wheel/disk, and maybe layers within chambers, depending on rate
\item Manually configured and operated, like an offline analysis
\item Data acquired before first collisions (trigger?)
\item Developer: Karoly Banicz
\end{itemize}

\item \textcolor{darkblue}{1~pb$^{-1}$ alignment}
\begin{itemize}
\item Aligning wheels/disks relative to tracker
\item Manually configured and operated, run on the CAF
\item Developer: Jim Pivarski
\end{itemize}

\item \textcolor{darkblue}{10~pb$^{-1}$ and 100~pb$^{-1}$ alignments}
\begin{itemize}
\item Aligning chambers relative to the tracker
\item Automated and parallelized, run on the CAF
\item (More detail on following pages)
\end{itemize}
\end{enumerate}
\end{frame}

\begin{frame}
\frametitle{Workflow for 10 and 100~pb$^{-1}$ alignments}
\begin{itemize}
\item 9 passes, the first having 15 iterations, the rest 5 (total of 55)
\begin{itemize}
\item First align wheels and disks, slow convergence
\item Next align innermost muon stations
\item Then fix innermost and align next, etc.
\item Then allow everything to float when close to optimum
\end{itemize}
\item Data from $W+Z$ and muons from QCD, controlled by a $p_T$ cut
and an ``outermost station residual'' cut (new code)
\end{itemize}

\textcolor{darkblue}{For 100~pb$^{-1}$:}
\begin{itemize}
\item Assuming all $W+Z$ and no QCD: 714,000 events
\item 54 CPUs + 1 control job
\item 11 min/iter $\to$ 10 hours wall time (3 min CPU time/iter\ldots)
\item 7 MB/iter output on {\it local} disk $\to$ 400 MB
\item Typical max memory: 300 MB, max swap: 700 MB
\end{itemize}
\end{frame}

\begin{frame}
\frametitle{Implementation details}
\begin{itemize}\setlength{\itemsep}{0.2 cm}
\item Single perl script creates all necessary directories and
configuration files; easy to stop and restart a half-finished job
\item One control job submits parallel sub-processes, waits for them to finish
\begin{itemize}
\item If this runs on the CAF, a CPU spends most of its time sleeping
\item Recently (last two days), a job running on CAF cannot successfully submit jobs
\end{itemize}

\item Muon geometry is passed from one iteration to the next via
SQLite files: last one copied into database by hand

\item Residuals monitoring through CommonAlignmentMonitor

\item Alignment monitoring through MuonGeometryIntoNtuples

\item For MC, combine datasets at the level of parameter matrices
(individual subjobs run on pure samples)
\end{itemize}
\end{frame}

\begin{frame}
\frametitle{Work to be done}
\textcolor{darkblue}{Cosmics/beam-halo chambers-on-disk, layers-in-chambers:}
\begin{itemize}
\item Understand pre-collisions trigger menu
\item Develop procedure on MC (beam-halo rate is ``factor-of-ten''
uncertain: level of feasibility will not be fully known)
\item Apply immediately to data as it becomes available (including
MTCC/GRE$x$ already available)
\item Our first priority is CSC alignment with beam-halo
\end{itemize}

\vfill
\textcolor{darkblue}{1~pb$^{-1}$ wheel/disk alignment to tracker:}
\begin{itemize}
\item Experience already shows this is feasible
\item Open question: will we be able to connect with chambers-on-disk
(above)?  Will $\vec{B}$ be ramped in between?
\item Study application of hardware alignment constraints using
survey-constraint mechanism
\end{itemize}
\end{frame}

\begin{frame}
\frametitle{Work to be done}
\textcolor{darkblue}{10/100~pb$^{-1}$ chambers to tracker:}
\begin{itemize}
\item Run procedure (very close: only CAF problems)
\item Submit new code, not strictly needed for 2\_0\_0 deadline, but we will aim for that
\begin{itemize}
\item Should the track-filter become an optional AlCaReco switch?  \\ I will present an implementation
\item Alternate track-refitter (not part of baseline procedure, but we want to test its performance)
\item All configuration and plotting scripts
\end{itemize}
\item Combine datasets, determine optimal cut and preferred parameters
for AlCaReco producer

(to be coordinated with MillePede muon alignment group)
\item Systematics studies
\begin{itemize}
\item Needs disk space!  400~MB $\times$ several configurations
\item Work in 1\_6\_7 with large data samples
\end{itemize}
\item Layer alignment feasibility study
\end{itemize}
\end{frame}

\begin{frame}
\frametitle{Conclusions}
\textcolor{darkblue}{Requests:}
\begin{itemize}
\item CAF/lxplus-accessible disk space (10 GB)
\item Submitting-from-CAF issue \\ (or ``long-running, mostly-sleeping job'' issue)
\item Occasional overnight use of 50 CPUs
\end{itemize}

\vfill
\textcolor{darkblue}{Very na\"ive question about CSA08:}
\begin{itemize}\setlength{\itemsep}{0.5 cm}
\item I don't think this is related to CSA08\ldots should it be?
\end{itemize}

\vfill
\textcolor{darkblue}{Status:}
\begin{itemize}
\item Baseline 100~pb$^{-1}$ procedure is almost in good shape: \\ early
tests show 300--800~$\mu$m, depending on station
\item Need to do controlled study with full workflow on CAF
\item Once that's in place, we focus attention on early alignment
\end{itemize}

\label{numpages}
\end{frame}

\end{document}
